\documentclass{article}

\usepackage{amsmath}
\usepackage{amssymb}
\usepackage{mathrsfs}
\newcommand{\uvec}[1]{\boldsymbol{\hat{\textbf{#1}}}}

\title{Summary of Physics equations}
\author{Me}
\date{}

\begin{document}
\maketitle

\section{Math}
Factorial
\begin{equation}
  n! = n(n-1)(n-2)...1 = \prod_{i=1}^{n}i
\end{equation}
Combinatory
\begin{equation}
  \binom{n}{i} = \frac{n!}{i!(n-i)!}
\end{equation}
Binomial Theorem
\begin{equation}
  (a+b)^n = \sum_{i=0}^n\binom{n}{i}a^ib^{n-i}
\end{equation}
Pascal Triangle\\\\
\begin{tabular}{rccccccccc}
$n=0$:&    &    &    &    &  1\\\noalign{\smallskip\smallskip}
$n=1$:&    &    &    &  1 &    &  1\\\noalign{\smallskip\smallskip}
$n=2$:&    &    &  1 &    &  2 &    &  1\\\noalign{\smallskip\smallskip}
$n=3$:&    &  1 &    &  3 &    &  3 &    &  1\\\noalign{\smallskip\smallskip}
$n=4$:&  1 &    &  4 &    &  6 &    &  4 &    &  1\\\noalign{\smallskip\smallskip}
\end{tabular}
\\\\
Property of Combinatory (deduced by using Pascal Triangle)
\begin{equation}
  \binom{n}{k} = \binom{n-1}{k-1} + \binom{n-1}{k}
\end{equation}

\section{Linear Algebra}
Vector ($v_i$ is the $i^{th}$ element of $\vec{v}$)
\begin{equation}
  \vec{v} = \left[v_i\right] = 
  \begin{bmatrix}
    v_1\\
    v_2\\
    \vdots\\
    v_n
  \end{bmatrix}
\end{equation}
Matrix
\begin{equation}
  A = \left[A_{ij}\right] = 
  \begin{bmatrix}
    a_{11} & a_{12} & \hdots & a_{1n}\\
    a_{21} & a_{22} & \hdots & a_{2n}\\
    \vdots & \ddots & \vdots & \vdots\\
    a_{m1} & a_{m2} & \hdots & a_{mn}\\
  \end{bmatrix}
\end{equation}
Norm of a vector (``length'' of the vector)
\begin{equation}
  \|\vec{u}\| = \sqrt{\sum_i u_i^2}
\end{equation}
Dot product ($\theta$ is the angle between $u$ and $v$)
\begin{equation}
  \vec{u} \cdot \vec{v} = \sum_i u_iv_i = \|\vec{u}\|\|\vec{v}\| \cos(\theta)
\end{equation}
Unit Vector
\begin{equation}
  \uvec{u} = \frac{\vec{u}}{\|\vec{u}\|}
\end{equation}
Cross product ($\uvec{i}$ is the unit vector in the x-axis,$\uvec{j}$ is the unit vector in the y-axis,$\uvec{k}$ is the unit vector in the z-axis)
\begin{equation}
  \vec{u} \times \vec{v} =
  \begin{vmatrix}
    \uvec{i} & \uvec{j} & \uvec{k}\\
    u_1 & u_2 & u_3\\
    v_1 & v_2 & v_3 
  \end{vmatrix}
\end{equation}
 Norm of the cross product ($\theta$ is the angle between $u$ and $v$)
\begin{equation}
  \|\vec{u} \times \vec{v}\| = \|\vec{u}\|\|\vec{v}\|\sin(\theta)
\end{equation}
Product Matrix - Vector
\begin{equation}
  \vec{v} = A\vec{u} \implies [v_i] = [\sum_k A_{ik}u_k]
\end{equation}
Product Matrix - Matrix
\begin{equation}
  A = BC \implies [A_{ij}] = [\sum_k B_{ik}C_{kj}]
\end{equation}
Triple scalar product (Proof by using a paralelepiped construct using the vectors $\vec{a}$,$\vec{b}$ and $\vec{c}$)
\begin{equation}
  \vec{a} \cdot(\vec{b} \times \vec{c}) =  \vec{c} \cdot(\vec{a} \times \vec{b}) =  \vec{b} \cdot(\vec{c} \times \vec{a})
\end{equation}
Triple cross product (This is by analyzing that $\vec{a} \times (\vec{b} \times \vec{c})$ is in the plane made by these two vectors $\vec{a}$ and $\vec{b}$)
\begin{equation}
  \vec{a} \times (\vec{b} \times \vec{c}) = (\vec{a}\cdot\vec{c}) \vec{b} - (\vec{a}\cdot\vec{b}) \vec{c}
\end{equation}
Line Equation ($\vec{\alpha}_0$ is a reference point to the line, $\uvec{u}$ is a unit vector in the direction of the line and $t$ is a just parameter)
\begin{equation}
  \vec{r} = \vec{\alpha}_0 + t\uvec{u}
\end{equation}
Plane Equation ($\vec{r}_0$ is a point of reference in the plane, $\uvec{n}$ is a unit vector normal to the plane )
\begin{equation}
  \uvec{n} \cdot (\vec{r} - \vec{r}_0) = \vec{0} 
\end{equation}

\section{Calculus}
Derivative
\begin{equation}
  \frac{df}{dx} = f'(x) = lim_{h \to 0}\frac{f(x+h) - f(x)}{h}
\end{equation}
Integrative
\begin{equation}
  F(x) = \int f(x) dx
\end{equation}
\begin{equation}
  f(x) = \frac{dF}{dx}
\end{equation}
Area (A) under the curve f from a to b
\begin{equation}
  A_{a \to b} = \int_a^b f(x) dx = F(b) - F(a)
\end{equation}
Rapid Proof
\begin{equation}
   f(a)h = A_{a \to a+h} = \int_a^{a+h} f(x) dx = F(a+h) - F(a)
\end{equation}
when $h \to 0$
\begin{equation}
  f(a) = \frac{F(a+h) - F(a)}{h} = \frac{dF}{dx}\Bigr|_{x=a} = F'(x=a) = f(a)
\end{equation}
Neperian number ($e$)
\begin{equation}
  e = lim_{n \to \infty}(1 + \frac{1}{n})^n
\end{equation}

\section{Multivariable Calculus}
Gradient of a function
\begin{equation}
  \nabla f(\vec{r}) = (\frac{\partial f}{\partial x},\frac{\partial f}{\partial y},\frac{\partial f}{\partial z})
\end{equation}
\begin{equation}
  df = \nabla f \cdot d\vec{r} = \frac{\partial f}{\partial x} dx + \frac{\partial f}{\partial y} dy + \frac{\partial f}{\partial z} dz 
\end{equation}
when $\vec{h} \to \vec{0}$
\begin{equation}
  df = f(\vec{r} + \vec{h}) - f(\vec{r}) =  \nabla f \cdot \vec{h} 
\end{equation}
Divergence
\begin{equation}
  \nabla \cdot \vec{F}(\vec{r}) = \frac{\partial F}{\partial x} +  \frac{\partial F}{\partial y} +  \frac{\partial F}{\partial z}
\end{equation}
Rotational
\begin{equation}
  \nabla \times \vec{F}(\vec{r}) =
  \begin{vmatrix}
    \uvec{i} & \uvec{j} & \uvec{k} \\
    \frac{\partial}{\partial x} & \frac{\partial}{\partial y} & \frac{\partial}{\partial z}\\
    F_x & F_y & F_z
  \end{vmatrix}
\end{equation}
Gauss Theorem
\begin{equation}
  \oint_{\partial V} \vec{F} \cdot d\vec{S} = \int \nabla \cdot \vec{F} dV 
\end{equation}
Stokes Theorem
\begin{equation}
  \oint_{\partial A} \vec{F} \cdot d\vec{l} = \int \nabla \times \vec{F} \cdot d\vec{S}
\end{equation}

\section{Complex Numbers}
Definition of complex unit ($i$ and sometimes $j$)
\begin{equation}
  i^2 = -1
\end{equation}
Euler's formula
\begin{equation}
  e^{i\theta} = \cos(\theta) + i\sin(\theta)
\end{equation}
Euler's identity
\begin{equation}
  e^{i\pi}+1=0
\end{equation}
Arithmetic with complex numbers ($z_1 = x_1 + iy_i$ and $z_2 = x_2 + iy_2$ with $x_1,y_1,x_2,y_2 \in \mathbb{R}$)
\begin{equation}
  \begin{split}
    z_1 \pm z_2  = (x_1 \pm x_2) + i(y_1 \pm y_2)\\
    z_1\times z_2 = (x_1x_2 - y_1y_2) + i(x_1y_2 + x_2y_1)
  \end{split}
\end{equation}
Poles of a function f(z) ($z_i$, in general it is a singularity)
\begin{equation}
  lim_{z \to z_i}f(z) \to \infty
\end{equation}
Residue Theorem
\begin{equation}
  \oint_{\mathscr{C}} f(z) dz = \frac{1}{2\pi i}(\sum_i lim_{z \to z_i} f(z)(z-z_i))
\end{equation}

\section{Statistics}
Mean
\begin{equation}
  \mu = \frac{\sum_{i=1}^n x_i}{n}
\end{equation}
Standard Deviation
\begin{equation}
  \sigma = \sqrt{\frac{\sum_{i=1}^n(x_i - \mu)^2}{n}}
\end{equation}
Normal Distribution
\begin{equation}
  f(x) = \frac{1}{\sqrt{2\pi \sigma^2}} e^{-\frac{(x-\mu)^2}{2\sigma^2}}
\end{equation}
Binomial Distribution ($p$ is the probability that the positive event happened, this distribution is used when there is just 2 possible outcomes)
\begin{equation}
  B(p,x) = \binom{n}{x}p^x(1-p)^{n-x}
\end{equation}
Poisson Distribution
\begin{equation}
  P_o(x) = \frac{e^{-\lambda}\lambda^x}{x!}
\end{equation}


\section{Probability}
Sample of Events
\begin{equation}
  S = \{E_1,E_2,E_3, \dots,E_n\}
\end{equation}
Probability of an event $E_i$ is $P(E_i)$
\begin{equation}
  \sum_i P(E_i) = 1
\end{equation}
In continuous
\begin{equation}
  \int_{\Omega} P(x) dx = 1
\end{equation}
Expected Value
\begin{equation}
  x = \sum_i x_i P(x_i)
\end{equation}
\begin{equation}
  x = \int_{\Omega}xP(x) dx
\end{equation}
Example: Dices - Expected value ($P(1) =P(2) = P(3) = P(4) = P(5) = P(6) = \frac{1}{6}$)
\begin{equation}
  1P(1) + 2P(2) + 3P(3) + 4P(4) + 5P(5) + 6P(6) = \frac{21}{6} = 3.5
\end{equation}
If A and B are indepents:
\begin{equation}
  P(A \cup B) = P(A) + P(B)
\end{equation}
If A and B are not indepents:
\begin{equation}
  P(A \cup B) = P(A) + P(B) - P(A\cap B)
\end{equation}
Conditional Probability (probability that A happens if B happened)
\begin{equation}
  P(A|B) = \frac{P(A\cap B)}{P(B)}
\end{equation}
Bayes Theorem
\begin{equation}
  P(A|B) = \frac{P(B|A)P(A)}{P(B)}
\end{equation}


\section{Transformations and Series}
All functions can be expressed as a polynomial of infinity degree (This is used to solve ordinary differential equations, just replace this in the equation and find the coefficients $a_i$)
\begin{equation}
  f(x) = \sum_{i=0}^{\infty} a_ix^i
\end{equation}
Taylor Series (expansion around $x_0$, $f^{i}$ is the $i^{th}$derivative of f)
\begin{equation}
  f(x) = \sum_{i=0}^{\infty}\frac{f^{i}(x_0)(x-x_0)^i}{i!}
\end{equation}
Laplace (To solve linear differential equations)
\begin{equation}
  F(p) = \mathscr{L}\{f(x)\} = \int_0^{\infty} f(x) e^{-px}dx
\end{equation}
Fourier Series for periodic functions (To solve partial differential equations)
\begin{equation}
  f(x) = a_0  + \sum_{n=1}^{\infty} a_n sin(\frac{2\pi n}{T}x) + b_n cos(\frac{2\pi n}{T}x)
\end{equation}
Fourier Transform
\begin{equation}
  F(w) = \mathscr{F}\{f(x)\} = \int_{-\infty}^{\infty}f(x)e^{-jwx} dx
\end{equation}
\begin{equation}
  f(x) = \mathscr{F}^{-1}\{F(w)\} = \int_{-\infty}^{\infty}F(w)e^{jwx} dx
\end{equation}

\section{Mechanics}
Linear Momentum definition
\begin{equation}
  \vec{P} = m\vec{v} = m\frac{d\vec{r}}{dt}
\end{equation}
Newton's First Law
\begin{equation}
  \sum \vec{F}_i^{ext} = \vec{0} \implies \sum \vec{P}_i^{sys} = constant
\end{equation}
Newton's Second Law
\begin{equation}
  \vec{F} = \frac{d\vec{P}}{dt} 
\end{equation}
\begin{equation}
  m = constant \implies \vec{F} = m\vec{a} = m\frac{d\vec{v}}{dt}= m\frac{d^2\vec{r}}{dt^2}
\end{equation}
Newton's Third Law
\begin{equation}
  \vec{F}_{ij} = - \vec{F}_{ij}
\end{equation}
Center of Mass
\begin{equation}
  \vec{r}_{CM} = \frac{\sum_i m_i\vec{r}_i}{\sum_i m_i}
\end{equation}
Rigid Body ($\vec{r'}_i$ is the position of any particle of the rigid body with respect of the center of mass)
\begin{equation}
  \vec{r}_i = \vec{r}_{CM} + \vec{r'}_i
\end{equation}
Since it is a rigid body then:
\begin{equation}
  \|\vec{r'}_i\| = constant
\end{equation}
Angular Momentum
\begin{equation}
  \begin{split}
    \vec{L} = \vec{r} \times \vec{P} =\vec{r} \times m\vec{v} = \vec{r} \times m(\vec{w} \times \vec{r}) = m(\vec{r}\cdot \vec{r}) \vec{w} = mr^2 \vec{w}
  \end{split}
\end{equation}
Inertia Moment (where $r_i$ is the perpendicular distance from the position of mass $m_i$ to the rotation axis)
\begin{equation}
  I = \sum_i m_ir_i^2
\end{equation}
\begin{equation}
  I = \int_V r^2dm
\end{equation}
Angular Momentum (General Definition)
\begin{equation}
  \vec{L} = I\vec{w} = I\frac{d\vec{\theta}}{dt}
\end{equation}
Angular Momentum Conservation
\begin{equation}
  \sum \vec{\tau}^{ext}_i = 0 \implies \sum \vec{L}^{sys}_i
\end{equation}
Torque
\begin{equation}
  \vec{\tau} = \frac{d\vec{L}}{dt} = \vec{r} \times \vec{F}
\end{equation}
Gravitation Law (Force exerted on $m_i$ by $m_j$)
\begin{equation}
  \vec{F}_g = -Gm_im_j\frac{\vec{r}_i - \vec{r}_j}{\|\vec{r}_i - \vec{r}_j\|^3}
\end{equation}
Work definition
\begin{equation}
  W_{\vec{r}_0 \implies \vec{r}_f} = \int_{\vec{r}_0}^{\vec{r}_f}\vec{F} \cdot d\vec{r}
\end{equation}
This lead to energy concept and energy conservation:
\begin{equation}
  m\vec{a} = \vec{F} \implies \int (m\vec{a} - \vec{F}) \cdot d\vec{r} = \int \vec{0} \cdot d\vec{r} = 0 
\end{equation}
Kinetic Energy
\begin{equation}
\Delta E_k = \int_{\vec{r}_i}^{\vec{r}_f}\vec{F} \cdot d\vec{r} = \int_{\vec{r}_i}^{\vec{r}_f}m\frac{d\vec{v}}{dt} \cdot d\vec{r} = \int_{v_i}^{v_f}m\vec{v} \cdot d\vec{v} = \frac{mv_f^2}{2} - \frac{mv_0^2}{2}
\end{equation}
\begin{equation}
  E_k = \frac{mv^2}{2}
\end{equation}
Potential Energy ($E_p$ or $U$)
\begin{equation}
  \Delta E_p = E_p(r_f) - E_p(r_0) =  \int_{\vec{r}_0}^{\vec{r}_f}\nabla U(\vec{r})\cdot d\vec{r}
\end{equation}
For the case of Gravitational forces ($\vec{r}_j = \vec{0}$ and $\vec{r}_i = \vec{r}$)
\begin{equation}
  \Delta E_p = \int_{\vec{r}_0}^{\vec{r}_f} -\vec{F} \cdot d\vec{r} = \int_{\vec{r}_0}^{\vec{r}_f} Gm_im_j\frac{\vec{r}}{\|\vec{r}\|^3} \cdot d\vec{r} = -\frac{Gm_im_j}{\|\vec{r}_f\|} + \frac{Gm_im_j}{\|\vec{r}_0\|} 
\end{equation}
In case of the gravitation force, the potential energy would be:
\begin{equation}
  U(\vec{r}) = -\frac{Gm_im_j}{\|\vec{r}\|} 
\end{equation}
Conservative Forces
\begin{equation}
  \vec{F}_c = - \nabla \phi
\end{equation}
Rotational Energy
\begin{equation}
  \begin{split}
    \Delta E_r = \int_{\vec{r}_0}^{\vec{r}_f}\vec{F} \cdot d\vec{r} = \int_{t_0}^{t_f}\vec{F} \cdot \vec{v}dt =   \int_{t_0}^{t_f} \vec{F} \cdot (\vec{r} \times \vec{w})dt = \\ \int_{t_0}^{t_f}(\vec{r} \times \vec{F}) \cdot \vec{w}dt = \int_{t_0}^{t_f} (\vec{r} \times \vec{F}) \cdot \vec{w}dt =  \int_{t_0}^{t_f} \vec{\tau}\cdot d\vec{\theta} = \int_{\theta_0}^{\theta_f}\tau d\theta
  \end{split}
\end{equation}
\begin{equation}
  \begin{split}
    \Delta E_r = \int_{t_0}^{t_f} \vec{\tau}\cdot d\vec{\theta} =  \int_{t_0}^{t_f} I\frac{d\vec{w}}{dt}\cdot d\vec{\theta} =\\  \int_{t_0}^{t_f} I \vec{w} \cdot d\vec{w} = \int_{w_0}^{w_f} Iwdw = \frac{Iw_f^2}{2} - \frac{Iw_0^2}{2}
  \end{split}
\end{equation}
\begin{equation}
  E_r = \frac{Iw^2}{2}
\end{equation}
Energy Conservation (When there is just conservative forces)
\begin{equation}
  \Delta E_k + \Delta E_p + \Delta E_r = 0
\end{equation}
\begin{equation}
  E_k + E_p + E_r = constant
\end{equation}
For Rigid Bodies - Kinetic Energy
\begin{equation}
  E_k = \sum_i \frac{m_iv_i^2}{2} = \sum_i \frac{m_i\|\dot{\vec{r}}_i^2\|}{2} = \sum_i \frac{m_i \dot{\vec{r}}_i \cdot \dot{\vec{r}}_i }{2} 
\end{equation}
Since $\vec{r}_i = \vec{r}_{CM} + \vec{r'}_i$ and $\vec{r'}_i \cdot \vec{r'}_i = constant \implies \vec{r'}_i \cdot \dot{\vec{r'}}_i = 0$ and also $M = \sum_im_i$:
\begin{equation}
  \vec{r'}_i \cdot \dot{\vec{r'}}_i = 0 \implies \dot{\vec{r'}}_i = \vec{w} \times \vec{r'}_i
\end{equation}
\begin{equation}
  \begin{split}
    E_k = \sum_i \frac{m_i \dot{\vec{r}}_i \cdot \dot{\vec{r}}_i }{2} = \sum_i \frac{m_i (\dot{\vec{r}}_{CM} + \dot{\vec{r'}}_i) \cdot (\dot{\vec{r}}_{CM} +\dot{\vec{r'}}_i)}{2} =  \sum_i \frac{m_i \dot{\vec{r}}_{CM} \cdot \dot{\vec{r}}_{CM}}{2} + \\\sum_i \frac{m_i \dot{\vec{r'}}_i \cdot \dot{\vec{r'}}_i}{2} + \vec{r}_{CM} \cdot \sum_i m_i \dot{\vec{r'}}_i =  \frac{M\dot{\vec{r}}_{CM} \cdot \dot{\vec{r}}_{CM}}{2} + \sum_i \frac{m_i \dot{\vec{r'}}_i \cdot \dot{\vec{r'}}_i}{2}
  \end{split}
\end{equation}
The first term of the previous equation is the kinetic energy of traslation
\begin{equation}
  K = \frac{M\vec{r}_{CM} \cdot \vec{r}_{CM}}{2}
\end{equation}
The second term can be viewd as ($d_i$ is the distance from that point to the axis of rotation):
\begin{equation}
  \begin{split}
   R = \sum_i \frac{m_i \dot{\vec{r'}}_i \cdot \dot{\vec{r'}}_i}{2} =  \sum_i \frac{m_i (\vec{w}\times\vec{r'}_i) \cdot \vec{w}\times\vec{r'}_i)}{2} = \sum_i \frac{m_i \vec{r'}_i \cdot (-\vec{w} \times (\vec{w} \times \vec{r'}_i))}{2} = \\\sum_i \frac{m_i \vec{r'}_i \cdot ((\vec{w}\cdot\vec{w}) \vec{r'}_i - (\vec{w}\cdot\vec{r'}_i) \vec{w})}{2} = w^2\sum_i \frac{m_id_i^2}{2} = \frac{Iw^2}{2} 
  \end{split}
\end{equation}
For rigid bodies ($K$ is the kinetic energy of traslation and $R$ is the rotational energy respect to an axis that passes the center of mass):
\begin{equation}
  E_k = K + R
\end{equation}
Wave Equation
\begin{equation}
  \frac{d^2y}{dx^2} + w^2y = 0
\end{equation}
Solution of the wave equation
\begin{equation}
    y = A_1cos(wx) + A_2sin(wx) =  Bcos(wx+ \phi)
\end{equation}
Spring (just in x axis, $k$: Spring constant)
\begin{equation}
  F(x) = -kx
\end{equation}
Spring - Movement equation
\begin{equation}
  F(x) = m\frac{d^2x}{dt^2} = -kx \implies \frac{d^2x}{dt^2} + \frac{k}{m}x = 0
\end{equation}
Spring - Solution ($w = \sqrt{\frac{k}{m}}$)
\begin{equation}
  x = Acos(wt + \phi)
\end{equation}
Pendulum simple (length $l$, mass $m$, gravity acceleration)
\begin{equation}
  \tau = \vec{r} \times \vec{F} = -mgl\sin(\theta)
\end{equation}
If $\theta \to 0$ $\implies$ $\sin(\theta) \sim \theta$
\begin{equation}
  \tau = -mgl\theta = \frac{dL}{dt} = \frac{d(ml^2w)}{dt} = ml^2\frac{d^2\theta}{dt^2}
\end{equation}
\begin{equation}
  \frac{d^2\theta}{dt^2} + \frac{g}{l}\theta = 0
\end{equation}
Solution ($w=\sqrt{\frac{g}{l}}$)
\begin{equation}
  \theta = A\cos(wt + \phi)
\end{equation}
Mechanical Wave in a string ($y = y(x,t)$ is the amplitude of the wave in the $x$ position at the time $t$) - Equation
\begin{equation}
  \frac{\partial^2y}{\partial t^2} = v^2\frac{\partial^2y}{\partial x^2}
\end{equation}
Solution of the wave equation ($w$ : angular frequency, $k$ : wave number)
\begin{equation}
  y = Acos(wt + kx + \phi)
\end{equation}
In general (3D case): Wave Equation
\begin{equation}
  \frac{\partial^2 F}{\partial t^2} = v^2\nabla^2 F
\end{equation}
Solution of the Wave Equation (3D) ($\vec{k}$ : Wave vector (points in the direction of the propagation of the wave), $w$ : angular frequency)
\begin{equation}
  F(\vec{r},t) = Acos(wt + \vec{k}\cdot\vec{r} + \phi )
\end{equation}
Lagrangian ($T$ : Kinetic Energy, $U$ : Potential Energy)
\begin{equation}
  L(q,\dot{q},t) = T(q,\dot{q},t) - U(q,\dot{q},t)
\end{equation}
Action
\begin{equation}
  S = \int_{t_0}^{t_f}Ldt
\end{equation}
Euler-Lagrange Equation (obatined when minimizing the action $S$)
\begin{equation}
  \frac{\partial L}{\partial q} - \frac{d}{dt}(\frac{\partial L}{\partial \dot{q}}) = 0
\end{equation}
Example of Euler-Lagrange Equation ($T$ = $\frac{m\dot{x}^2}{2}$, $U$ = $mgx$)
\begin{equation}
  \frac{\partial L}{\partial q} - \frac{d}{dt}(\frac{\partial L}{\partial \dot{q}}) = -mg -m\ddot x=0 \implies a = -g
\end{equation}

\section{Fluid Mechanics}
Bernoulli Equations (which is just conservation of energy)
\begin{equation}
  \rho gh + \frac{\rho v^2}{2} + p = constant
\end{equation}
Navier-Stokes Equation ($\vec{v} = \vec{v}(\vec{r},t)$ is the velocity of the fluid at the position $\vec{r}$ and the time $t$)
\begin{equation}
  \rho \frac{D \vec{v}}{Dt} = \rho (\frac{\partial \vec{v}}{\partial t} + (\vec{v}\cdot \nabla)\vec{v}) = -\nabla p+\rho \vec{g} + \mu \nabla^2 \vec{v}
\end{equation}
Continuity Equation
\begin{equation}
  \nabla \cdot \vec{v} = \vec{0}
\end{equation}

\section{Thermodynamics}
Pressure ($F$ is the magnitude of the force perpendicular to $A$)
\begin{equation}
  P = \frac{F}{A}
\end{equation}
Ideal Gas Law ($P$ : pressure, $V$ : Volume, $n$ : Number of moles, $R$ : Ideal gas constant, $T$ : Temperature)
\begin{equation}
  PV = nRT
\end{equation}
Work done by the system
\begin{equation}
  W = \int_{V_0}^{V_f} P dV
\end{equation}
First Law of Thermodynamics ($Q$ is heat, $W$ is work and $\Delta U$ is the change of internal energy), this law meas conservation of energy
\begin{equation}
  Q = W + \Delta U
\end{equation}
Second Law of thermodynamics ($\Delta S$ is change of entropy)
\begin{equation}
  \Delta S \ge 0
\end{equation}
Isothermic Process ($T = constant$)
\begin{equation}
  W = \int_{V_0}^{V_f} P dV =  \int_{V_0}^{V_f} \frac{nRT}{V} dV = nRT \ln(\frac{V_f}{V_0})
\end{equation}
Since $U=U(T)$, then for $T=constant \implies \Delta U = 0$
\begin{equation}
  Q = nRT \ln(\frac{V_f}{V_0})
\end{equation}
Isobaric Process ($P = P_0 = constant$)
\begin{equation}
  W = \int_{V_0}^{V_f} P dV = P_0(V_f - V_0)
\end{equation}
In ideal case: $\Delta U(T) = C_P\Delta T$, where $C_P$ is the Calorific Capacity at constant pressure 
\begin{equation}
  Q = P_0(V_f - V_0) + C_P(T_f - T_0)
\end{equation}
Isocoric Process ($V = V_0 = constant$)
\begin{equation}
  W = \int_{V_0}^{V_f} P dV = 0
\end{equation}
In ideal case: $\Delta U(T) = C_V\Delta T$, where $C_V$ is the Calorific Capacity at constant volume
\begin{equation}
  Q = C_V\Delta T = C_V(T_f - T_0)
\end{equation}
Heat Capacity ($Q$ is heat)
\begin{equation}
  Q = \int_V C dT
\end{equation}
Heat Transfer by convection ($A$ is the area of the surface, $T_f$ is the fluid temperature (like air for example), $T$ is the temperature of the surface and $h$ is the convection constant)
\begin{equation}
  \dot{Q}(t) = hA(T_f - T)
\end{equation}
Heat Transfer by conduction (Fourier's Law, $q$ is the heat flux)
\begin{equation}
  q_x = -K\frac{dT}{dx}
\end{equation}
\begin{equation}
  \vec{q} = -K\nabla T
\end{equation}
Diffusion Equation
\begin{equation}
  \frac{\partial T}{\partial t} = \kappa \nabla^2T
\end{equation}

\section{Electromagnetism}
Current definition
\begin{equation}
  I = \frac{dq}{dt}
\end{equation}
Current density definition ($A$ is the transversal area through which I passes)
\begin{equation}
  J = \frac{I}{A}
\end{equation}
\begin{equation}
  I = \int_{A} \vec{J} \cdot d\vec{S}
\end{equation}
Coulomb's Law
\begin{equation}
    \vec{F}_E = kq_iq_j\frac{\vec{r}_i - \vec{r}_j}{\|\vec{r}_i - \vec{r}_j\|^3}
\end{equation}
Electric Field
\begin{equation}
  \vec{E} = lim_{q_i \to 0} \frac{\vec{F}_e}{q_i} = kq_j\frac{\vec{r}_i - \vec{r}_j}{\|\vec{r}_i - \vec{r}_j\|^3}
\end{equation}
Electric Force
\begin{equation}
  \vec{F}_E = q\vec{E}
\end{equation}
Magnetic Force
\begin{equation}
  \vec{F_B} = q\vec{v} \times \vec{B}
\end{equation}
\begin{equation}
  \vec{F_B} = \int Id\vec{l} \times \vec{B}
\end{equation}
\begin{equation}
  \vec{F_B} = \int_{V} \vec{J} \times \vec{B} dV
\end{equation}
Magnetic Field
\begin{equation}
  \vec{B} = \int \frac{\mu_0Id\vec{l} \times \vec{r}}{4\pi\|\vec{r}\|^3}
\end{equation}
\begin{equation}
  \vec{B} = \int \frac{\mu_0\vec{J} \times \vec{r} dV}{4\pi\|\vec{r}\|^3}
\end{equation}
Gauss Law for Electric Field and Magnetic Field ($\rho = \frac{dq}{dV}$ is charge density)
\begin{equation}
  \oint_{\partial V}\vec{E} \cdot d\vec{S} = \frac{Q}{\epsilon_0}
\end{equation}
\begin{equation}
  \nabla \cdot \vec{E} = \frac{\rho}{\epsilon_0}
\end{equation}
\begin{equation}
  \oint_{\partial V}\vec{B} \cdot d\vec{S} = 0
\end{equation}
\begin{equation}
  \nabla \cdot \vec{B} = 0
\end{equation}
Ampere's Law
\begin{equation}
  \oint_{\partial A} \vec{B} \cdot d\vec{l} = \mu_0I
\end{equation}
\begin{equation}
  \nabla \times \vec{B} = \mu_0 \vec{J}
\end{equation}
Ampere's Law (with the displacement current)
\begin{equation}
  \oint_{\partial A} \vec{B} \cdot d\vec{l} = \mu_0I + \mu_0\epsilon_0\frac{d\int_{\partial V}\vec{E} \cdot d\vec{S}}{dt}
\end{equation}
\begin{equation}
  \nabla \times \vec{B} = \mu_0 \vec{J} + \mu_0\epsilon_0\frac{\partial \vec{E}}{\partial t}
\end{equation}
Faraday's Law
\begin{equation}
 \oint_{\partial A} \vec{E} \cdot d\vec{l} = -\frac{d\int_{\partial V}\vec{B} \cdot d\vec{S}}{dt}  
\end{equation}
\begin{equation}
  \nabla \times \vec{E} = -\frac{\partial \vec{B}}{\partial t}
\end{equation}
Electromagnetic Waves in vaccum ($\rho = 0$ (no charge), $\vec{J} = \vec{0}$ (no current))
\begin{equation}
  \nabla \times \nabla \times \vec{E} = -\frac{\partial \nabla \times \vec{B}}{\partial t} = -\mu_0\epsilon_0\frac{\partial^2\vec{E}}{\partial t^2}
\end{equation}
Since $\nabla \times \nabla \times \vec{E} = \nabla(\nabla \cdot \vec{E}) -\nabla^2 \vec{E}$:
\begin{equation}
  \nabla^2 \vec{E} = \mu_0\epsilon_0\frac{\partial^2\vec{E}}{\partial t^2}
\end{equation}
Since $c = \frac{1}{\sqrt{\mu_0\epsilon_0}}$
\begin{equation}
    \nabla^2 \vec{E} = \frac{1}{c^2}\frac{\partial^2\vec{E}}{\partial t^2}
\end{equation}
In the same way:
\begin{equation}
  \nabla \times \nabla \times \vec{B} = \mu_0\epsilon_0\frac{\partial \nabla \times \vec{E}}{\partial t} = - \mu_0\epsilon_0 \frac{\partial^2 \vec{B}}{\partial t^2}
\end{equation}
Since $\nabla \times \nabla \times \vec{B} = \nabla(\nabla \cdot \vec{B}) -\nabla^2 \vec{B}$:
\begin{equation}
  \nabla^2 \vec{B} = \mu_0\epsilon_0\frac{\partial^2\vec{B}}{\partial t^2}
\end{equation}
Solution of the wave equation mentioned before (where $c=wk$, $w$ : angular frequency and $k$ : wave number)
\begin{equation}
  \vec{E}(\vec{r},t) = E_0cos(wt + \vec{k} \cdot \vec{r} + \phi)
\end{equation}
\begin{equation}
  \vec{B}(\vec{r},t) = B_0cos(wt + \vec{k} \cdot \vec{r} + \phi)
\end{equation}

\section{Quantum Mechanics}
Wave Function $\psi(\vec{r})$:
\begin{equation}
  \int_V \psi^*(\vec{r}) \psi(\vec{r})d\vec{r} = 1
\end{equation}
Probability to find the particle in the region $\Omega$:
\begin{equation}
  P(\Omega) = \int_\Omega \psi^*(\vec{r}) \psi(\vec{r}) d\vec{r}
\end{equation}
Position Operator
\begin{equation}
  \widehat{\vec{r}}\psi(\vec{r}) = \vec{r}\psi(\vec{r})
\end{equation}
Expected position
\begin{equation}
  \vec{R} = \int_V \psi^*(\vec{r}) \vec{r}\psi(\vec{r}) d\vec{r}
\end{equation}
Momentum Operator
\begin{equation}
  \widehat{P} = \frac{\hbar}{i}\nabla
\end{equation}
Schrodinguer Equation
\begin{equation}
  \widehat{H}\psi = \frac{\widehat{P}^2}{2m}\psi + \widehat{V}\psi = -\frac{\hbar^2}{2m}\nabla^2 \psi + V(\vec{r})\psi =  E\psi
\end{equation}

\section{Einstein Relativity}
Time dilatation ($t$ is the time in a reference frame which is in rest and $t'$ is the time in a reference frame which is moving with constant velocity $v$, all of this in the x axis). (The idea is to decipher this is just using a mirror in the roof and the soil of a train moving and a ray of light bouncing between those mirrors, also using the idea that $c$ is the same value in all reference systems)
\begin{equation}
  \Delta t' = \Delta t \sqrt{1 - \frac{v^2}{c^2}}
\end{equation}
Length Contraction (by measuring using a laser, in other words using a ray of light)
\begin{equation}
  L' = c\Delta t' = c\Delta t \sqrt{1 - \frac{v^2}{c^2}} = L\sqrt{1 - \frac{v^2}{c^2}}
\end{equation}
Lorentz Transformations
\begin{equation}
  \begin{split}
    x = \frac{x' + vt'}{\sqrt{1 - \frac{v^2}{c^2}}}\\
    x' = \frac{x' - vt'}{\sqrt{1 - \frac{v^2}{c^2}}}\\
      t = \frac{t' + x'\frac{v}{c^2}}{\sqrt{1 - \frac{v^2}{c^2}}}\\
      t' = \frac{t - x\frac{v}{c^2}}{\sqrt{1 - \frac{v^2}{c^2}}}
  \end{split}
\end{equation}
Lorentz Constant
\begin{equation}
  \gamma = \frac{1}{\sqrt{1-\frac{v^2}{c^2}}}
\end{equation}
In general:
\begin{equation}
  \gamma(\vec{v}) = \frac{1}{\sqrt{1-\frac{\vec{v}\cdot \vec{v}}{c^2}}}
\end{equation}
In order to generalize Lorentz Transformations to any direction, we can use projections of the vector position along the axis of movement ($\vec{r'} \cdot \vec{v}$). So $\vec{r'} = \vec{r'}_1 + \vec{r'}_2$ where $\vec{r'}_1$ is parallel to $\vec{v}$ and $\vec{r'}_2$ is orthogonal to $\vec{v}$
\begin{equation}
  \vec{r'}_1 = (\vec{r'} \cdot \frac{\vec{v}}{\|\vec{v}\|} )\frac{\vec{v}}{\|\vec{v}\|}
\end{equation}
\begin{equation}
  \vec{r'}_2 = \vec{r'} - \vec{r'}_1
\end{equation}
From both equations about we get that (since we just get contraction length in the direction of $\vec{v}$)
\begin{equation}
  \begin{split}
    \vec{r} = \vec{v}t + \vec{r'}_2 + \frac{\vec{r'}_1}{\sqrt{1-\frac{\vec{v}\cdot \vec{v}}{c^2}}} = \vec{v}t + \vec{r'} - \vec{r'}_1 + \frac{\vec{r'}_1}{\sqrt{1-\frac{\vec{v}\cdot \vec{v}}{c^2}}} =  \vec{v}t + \vec{r'} - \vec{r'}_1(1-\gamma) =\\ \vec{r'} + \vec{v}\gamma t' - (\vec{r'} \cdot \frac{\vec{v}}{\|\vec{v}\|} )\frac{\vec{v}}{\|\vec{v}\|}(1 - \gamma) = \vec{r'} + (\gamma t' + \frac{\vec{r'}\cdot \vec{v}}{\vec{v} \cdot \vec{v}} (\gamma - 1) )\vec{v}
    \end{split}
\end{equation}
In the same way for time would be:
\begin{equation}
  t = \frac{t' + \frac{\vec{r'}\cdot v}{c^2}}{\sqrt{1 - \frac{v^2}{c^2}}} =\gamma(t' + \frac{\vec{r'}\cdot v}{c^2}) \\
\end{equation}
And also, due to the invariance:
\begin{equation}
  \vec{r'} = \vec{r} - (\gamma t - \frac{\vec{r}\cdot \vec{v}}{\vec{v} \cdot \vec{v}} (\gamma - 1) )\vec{v}
\end{equation}
\begin{equation}
  t' = \gamma(t - \frac{\vec{r}\cdot v}{c^2})
\end{equation}
From the previous equations we obtain that:
\begin{equation}
  d\vec{r} \cdot d\vec{r} = d\vec{r'} \cdot d\vec{r'} - c^2dt'^2 + c^2dt^2
\end{equation}
The previous equations can be writen as follows (which is called the invariance under Lorentz Transformation):
\begin{equation}
  c^2dt^2 - dx^2 - dy^2 -dz^2 = c^2dt'^2 - dx'^2 - dy'^2 - dz'^2
\end{equation}
If the frame reference we choose is situated in the particle that is moving (which means that $dx' = dy' = dz' = 0$), then:
\begin{equation}
  c^2dt^2 - dx^2 - dy^2 -dz^2 = c^2dt'^2 = c^2d\tau^2
\end{equation}
We change $t'$ by $\tau$ because this is a special time called ``proper time'' (because is the time measured by the moving particle). Now let's find the four-position and four-velocity of a moving particle.
\begin{equation}
  \vec{R} = (ct,\vec{r}) = (ct,x,y,z)
\end{equation}
Introducing the inner product that we will use in this space (Minkowsi):
\begin{equation}
  d\vec{R} \cdot d\vec{R} = cdt^2 - d\vec{r}\cdot d\vec{r} = cdt^2 - dx^2 - dy^2 - dz^2 
\end{equation}
Now let's find the speed (Remember that $\frac{dt}{d\tau} = \gamma$):
\begin{equation}
  \vec{U} = \frac{d}{d\tau} \vec{R} = (c \gamma,\frac{d\vec{r}}{d\tau}) = (c \gamma,\frac{d\vec{r}}{dt} \frac{dt}{d \tau}) = (c \gamma,\frac{d\vec{r}}{dt} \gamma)= \gamma (c,\vec{v})
\end{equation}
where $\vec{v}$ is the velocity of the particle with respect to the frame reference in rest. We can check that $\vec{U} \cdot \vec{U}$ is an invariant under any frame reference.
\begin{equation}
  \vec{U} \cdot \vec{U} = \gamma^2(c^2 - \vec{v} \cdot \vec{v}) = c^2
\end{equation}
Now let's find the four-momentum:
\begin{equation}
  \vec{P} = m\vec{U} = (\gamma mc,\gamma m\vec{v}) = (\frac{E}{c},\frac{mv}{\sqrt{1 - \frac{v^2}{c^2}}}) = (\gamma mc, \vec{p})
\end{equation}
\end{document}
